\abschnitt{Discussion}
two distinc use-cases for \sfcoros:
\begin{itemize}
    \item replacement of \slcoros if suspention from within nested \cstack is
          required
    \item in context of cooperativly scheduled tasks
\end{itemize}

Creating \fibers/\uthreads requires first-class objects and \sycoros in order
to execute tasks cooperativly (topic of another proposal).\\
For the first use-case the author suggest to use an N4134-like syntax for
\sfcoros\ref{appendix}.\\
The similarities between \sfcoros and \slcoros are that both have to
preserve/restore callee-saved registers as well as to preserve/restore
the instruction pointer (return address).\\
Unlike \sless, \sfcoro are required to preserve/restore the stack pointer.

\uabschnitt{Unified syntax:}
In the context of clean code a \sfull N4134-syntax would be preferable - the
characteristics for \sfcoros based on N4134-syntax require:
\begin{itemize}
    \item stackfull
    \item asymmetric: two operations - \await and \yield - for context switch
    \item no first-class objects (coroutine structure/object is hidden by the
          compiler)
\end{itemize}

\uabschnitt{Indicator:}
Compiler needs indicator to know if s\sless or \sfull context switch is desired.\\
Introduction of new annotation \sfanno similar to \resumable as in previous versions
of N4134 proposal. The new annotation tells the compiler to execute the function in
an \sfcoro if it was invoked with \await. Otherwise it is called as ordinary function.\\
\cppf{annotation.cpp}

\uabschnitt{Stack type/size:}
Using \sfcoros requires to estimate the required stacksize. The annotation could accept
an type indicating which type iand size of stack should be created.\\
\cppf{fixed.cpp}
\cppf{segmented.cpp}
\cppf{configured.cpp}
