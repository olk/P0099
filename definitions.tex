\abschnitt{Definitions}
\uabschnitt{coroutine:}
enables explicit suspend and resume of its progress via additional operations by
preserving execution state and thus provides an enhanced control flow.\\
Coroutines have following characteristics\cite{N3985}:
\begin{itemize}
    \item values of local data persist between successive context switches
    \item execution is suspended as control leaves coroutine and resumed at
          certain time later
    \item symmetric or asymmetric control-transfer mechanism
    \item first-class object or compiler-internal structure
    \item stackful or stackless
\end{itemize}

\uabschnitt{\ascoro:}
provides two disting operations for contexts switch - one operation to
resume and one operation to suspend the coroutine.\\
A \ascoro is tightly coupled with its caller, e.g. on suspending the
coroutine transferes the execution control back to its invoker.

\uabschnitt{\sycoro:}
only one operation to resume/suspend the context is available.\\
A \sycoro does not know its caller, e.g. the execution control can passed to
any other \sycoro.

\uabschnitt{fiber/user-mode thread:}
fibers execute tasks in an co-operative multitasking environment involving
a scheduler\\.

\uabschnitt{segmented stack\cite{llvmseg}:}
also known as \textit{split stack}\cite{gccsplit}, represents a stack with
a non-contigous address-range.

\uabschnitt{\resumfn of N4134:}
is intented to provide an efficient language supported mechanism for \slcoros
introduceing two new keywords - \await and \yield.\\
Characteristics of N4134 \slcoros:
\begin{itemize}
    \item stackless
    \item asymmetric (suspend operations returns to caller of \slcoro)
    \item no first-class objects (coroutine structure/object is hidden by the
          compiler)
    \item uncommon implicit \textit{return}-statement\cite{N4134}
          \cppf{N4134/fib.cpp}
\end{itemize}
