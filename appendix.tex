\abschnitt{A. \textit{jump}-operation for SYSV ABI on x86\_64}\label{appendix}

The assembler code (from \boostcontext) shows how the \textit{jump}-operation
might be look like for SYSV ABI on x86\_64.
\asmf{jump.S}
Register \asm{rdi} contains a reference to the \cblock of the current execution
context \textit{X} and register \asm{rsi} points to the \cblock of the execution
context \textit{Y} which has to be resumed.\\
\newline
In lines 2-7 the content of the current non-volatile registers are stored in the
\cblock of \textit{X}.\\
Line 9 calculates the stack-pointer writes it to the \cblock in line 10.\\
Lines 11 and 12 do the same for the return address (will be assigned to the
instruction-pointer \asm{rip}).\\
\newline
The next block (lines 14-19) restore the content of non-volatile register for
the execution context \textit{Y}.\\
The stack-pointer is restored in line 21.\\
Line 22 moves the address of the instruction which should be executed in
\textit{Y} to register \asm{rcx}.\\
\newline
Lines 24 and 25 allow to transfer data (as return value in \textit{Y}) between
context jumps.\\
\newline
The next line transfers execution control (\textit{branch-and-link}) to
\textit{Y} by executing the instruction stored in \asm{rcx}.
