\abschnitt{A. \textit{jump}-operation for SYSV ABI on x86\_64}\label{appendix}

The assembler code (from \boostcontext) shows what the \textit{jump}-operation
might look like for SYSV ABI on x86\_64.
\asmf{jump.S}

Register \asm{rdi} takes the stack-address of to the current execution context
\textit{X} (containing the \cblock) and register \asm{rsi} contains the
stack-address of the target execution context \textit{Y} (containing the
\cblock) to be resumed.\\
\newline
In lines 10-15 the contents of the current non-volatile registers are pushed on
the \stack of \textit{X}.\\
\newline
Lines 19 and 21 exchange the stack-pointers - the current stack-pointer is
stored in and returned via the first argument (\asm{rdi}). The address of the
other context, contained in \asm{rsi}, is assigned to the stack-pointer
\asm{rsp}.\\
\newline
The next block (lines 24-29) restores the contents of non-volatile registers for
the execution context \textit{Y}.\\
\newline
Line 32 pops the address of the instruction which should be executed in
\textit{Y} to register \asm{r8}.\\
\newline
Lines 35 and 36 allow to transfer data (as return value in \textit{Y}) between
context jumps.\\
\newline
The next line transfers execution control (\textit{branch-and-link}) to
\textit{Y} by executing the instruction to which \asm{r8} points.
