%//////////////////////////////////////////////////////////////////////////////

\documentclass[paper=A4,pagesize,DIV=15]{scrartcl}

\usepackage[T1]{fontenc}
\usepackage[latin1]{inputenc}
\usepackage[british]{babel}

%\usepackage{fixltx2e}
\usepackage{ellipsis}
\usepackage{ragged2e}
\usepackage[final]{microtype}

\usepackage{palatino}

\usepackage{overcite}
\usepackage{booktabs}
\usepackage{fancyhdr}
\usepackage{listings}
\usepackage{perpage}
\usepackage{rotating}
\usepackage{svg}
\usepackage{tikz}
\usetikzlibrary{arrows,automata}
\usepackage{xcolor}
\usepackage{xspace}
\usepackage[colorlinks=true,
            urlcolor=blue,
            pdftex,
            pdfsubject  = {},
            pdfauthor   = {Oliver Kowalke, Nat Goodspeed},
            pdfkeywords = {C++,execution,context,coroutine,P0099},
            pdftitle    = {A low-level API for stackful context switching}]{hyperref}

%//////////////////////////////////////////////////////////////////////////////

\setlength{\parindent}{0pt} 

\makeatletter
    \renewcommand*\l@subsection{\@dottedtocline{2}{0em}{2.3em}}
    \renewcommand*\l@subsection{\@dottedtocline{3}{0em}{3.2em}}
    \renewcommand{\tableofcontents}{%
        \@starttoc{toc}
    }
\makeatother

%\renewcommand{\thesubsection}{\Roman{subsection}}

\newcommand{\pdfimg}[1]{\pdfximage{pics/#1}\pdfrefximage\pdflastximage}
\newcommand{\img}[1]{\mbox{\pdfimg{#1}}}
\newcommand{\imgc}[1]{\begin{center}\img{#1}\end{center}}

\newcommand{\cpp}[1]{
    \lstinline[
        language=C++,
        basicstyle=\ttfamily\color{black},
        keywordstyle=\color{blue},
        commentstyle=\color{green},
        stringstyle=\color{red}
    ] {#1}
}
\newcommand{\cppf}[1]{
    \lstinputlisting[
        language=C++,
        basicstyle=\ttfamily\color{black},
        keywordstyle=\color{blue},
        commentstyle=\color{red},
        stringstyle=\color{green}
    ] {code/#1}
}

\newcommand{\corobool}{\cpp{std::coroutine<>::operator safe_bool()}\xspace}
\newcommand{\coro}{\cpp{std::coroutine<>}\xspace}
\newcommand{\coroget}{\cpp{std::coroutine<>::get()}\xspace}
\newcommand{\coroiterator}{\cpp{std::coroutine<>::iterator}\xspace}
\newcommand{\coroop}{\cpp{std::coroutine<>::operator()(Args...)}\xspace}
\newcommand{\pullcorobool}{\cpp{std::pull_coroutine<>::operator safe_bool()}\xspace}
\newcommand{\pullcoro}{\cpp{std::pull_coroutine<>}\xspace}
\newcommand{\pullcoroget}{\cpp{std::pull_coroutine<>::get()}\xspace}
\newcommand{\pullcoroop}{\cpp{std::pull_coroutine<>::operator()()}\xspace}
\newcommand{\pushcoro}{\cpp{std::push_coroutine<>}\xspace}
\newcommand{\pushcoroop}{\cpp{std::push_coroutine<>::operator()(Arg)}\xspace}
\newcommand{\tuple}{\cpp{std::tuple<>}\xspace}

\newcommand{\await}{\textit{await}\xspace}
\newcommand{\csharp}{\textit{C#}\xspace}
\newcommand{\cblock}{\textit{control-block}\xspace}
\newcommand{\continuation}{\textit{continuation}\xspace}
\newcommand{\coopmultitasking}{\textit{cooperative multitasking}\xspace}
\newcommand{\corofunction}{\textit{coroutine-function}\xspace}
\newcommand{\escre}{\textit{escape-and-reenter}\xspace}
\newcommand{\escreops}{\textit{\escre~operations}\xspace}
\newcommand{\escreccomps}{\textit{\escre~recursive computations}\xspace}
\newcommand{\escrecloops}{\textit{\escre~loops}\xspace}
\newcommand{\resumfn}{\textit{resumeable function}\xspace}
\newcommand{\asyncops}{\textit{asynchronous-operations}\xspace}
\newcommand{\asyncres}{\textit{asynchronous-result}\xspace}

\newcommand{\awaitemu}{await_emu\cite{awaite}\xspace}
\newcommand{\boostasio}{boost.asio\cite{asio154}\xspace}
\newcommand{\boostcontext}{boost.context\cite{context154}\xspace}
\newcommand{\boostcoroutine}{boost.coroutine\cite{coroutine154}\xspace}
\newcommand{\boostcorosum}{boost.coroutine (Google Summer of Code 2006)\cite{coroutineSum2006}\xspace}

\newcommand{\abschnitt}[1]{
    \addcontentsline{toc}{subsection}{#1}
    \subsection*{#1}
}


%//////////////////////////////////////////////////////////////////////////////

\begin{document}
\small
\begin{tabbing}
    Document number: \= P0099R0\\
    Supersedes:      \> N4397\\
    Date:            \> 2015-09-25\\
    Project:         \> WG21, SG1\\
    Author:          \> Oliver Kowalke (oliver.kowalke@gmail.com), Nat Goodspeed (nat@lindenlab.com)\\
\end{tabbing}

\section*{A low-level API for stackful context switching}

%//////////////////////////////////////////////////////////////////////////////

\tableofcontents

%//////////////////////////////////////////////////////////////////////////////

\paragraph*{Revision History}
This document supersedes N4397. It elaborates a low-level API for stackful
context switching.\\
\newline
Changes since N4397:

\begin{itemize}
    \item \cpp{std::execution\_context} presented as pure library facility.
    \item Motivation section added.
    \item \cpp{std::execution\_context} no longer tracks parent relationships.
    \item Removed \cpp{operator bool()} and \cpp{operator!()}.
\end{itemize}

\abschnitt{Abstract}
This paper proposes a \emph{low-level} API for a stackful execution context, suitable
to act as building-block for high-level constructs like stackful coroutines from
N3985\cite{N3985} (boost.coroutine2\cite{bcoroutine2}) as well as to implement
effective cooperative multitasking (boost.fiber\cite{bfiber}).\\
Based on the proposed low-level API, the follow-up proposal N4398\cite{N4398}
introduces a unified syntax for stackless and stackful coroutines.\\
\newline
The most important features are:
\begin{itemize}
    \item first-class object that can be stored in variables or containers
    \item introduction of new keyword \resumable together with a lambda-like
          syntax
    \item symmetric transfer of execution control, i.e. suspend-by-call -
          enables a richer set of control flows than asymmetric transfer of
          comtrol (i.e. suspend-by-return as described in N4134)
    \item benefits of traditional stack management retained
    \item ordinary function calls and returns not affected
    \item working reference implementation in boost.context\cite{bcontext}
\end{itemize}

\abschnitt{Motivation}

This proposal refers to \boostcoroutine as reference implementation - providing
a test suite and examples (some are described in this section).\\
\newline
In order to support a broad range of execution control behaviour \pushcoro and\\
\pullcoro can be used to \escrecloops, to \escreccomps~and for \coopmultitasking
helping to solve problems in a much simpler and more elegant way than with only
a single flow of control.

\subsubsection*{'same fringe' problem}
The advantages can be seen particularly clearly with the use of a recursive
function, such as traversal of trees.\\
If traversing two different trees in the same deterministic order produces the
same list of leaf nodes, then both trees have the same fringe even if the tree
structure is different.\\
\newline
The same fringe problem could be solved using coroutines by iterating over the
leaf nodes and comparing this sequence via \cpp{std::equal()}. The range of data
values is generated by function \cpp{traverse()} which recursively traverses the
tree and passes each node's data value to its \pushcoro.\\
\pushcoro suspends the recursive computation and transfers the data value to
the main execution context.\\
\pullcoroiterator, created from \pullcoro, steps over those data values and
delivers them to \cpp{std::equal()} for comparison. Each increment of \pullcoroiterator
resumes \cpp{traverse()}. Upon return from \cpp{iterator::operator++()}, either
a new data value is available, or tree traversal is finished (iterator is
invalidated).
\cppf{same_fringe.cpp}

\subsubsection*{asynchronous operations with boost.asio}
In the past the code using asio's \asyncops was scattered by callbacks.
\boostasio provides with its new \asyncres feature a new way to simplify the
code and make it easier to read.\\
Proposal 'N3747: A Universal Model for Asynchronous Operations'\cite{n3747}
describes the usage of coroutines in the context of asynchronous operations.\\
\yieldcontext uses internally \boostcoroutine:
\cppf{asyncres.cpp} 

\subsubsection*{\csharp \await}
\csharp contains the two keywords \async and \await. \async introduces a
control flow that involves awaiting asynchronous operations. The compiler
reorganizes the code into a continuation-passing style. \await wraps the rest
of the function after calling \await into a continuation if the asynchronous
operation has not yet completed.\\
The project \awaitemu uses \boostcoroutine for a proof-of-concept
demonstrating the implementation of a full emulation of \csharp \await as a
library extension. Because of stackful coroutines \await is \textbf{not limited}
by "one level" as in \csharp.\\
Evgeny Panasyuk describes the advantages of \boostcoroutine over \await at
\channelnine.
\cppf{await.cpp}

\abschnitt{Background}
At the meeting in Urbana the committee accepted that stackless and stackful
coroutines are distinct and decided to pursue both kinds of coroutines.\\
This paper proposes a low-level API for context switching suitable to act as 
building-block for high-level constructs like stackful coroutines from
N3985\cite{N3985} (boost.coroutine2\cite{bcoroutine2}) or cooperative
multitasking (boost.fiber\cite{bfiber}).\\
The author evaluates an alternative, 'resumable lambda'-like syntax for
stackful context switching.

\abschnitt{Definitions}

\uabschnitt{execution context:}
environment where program logic is evaluated in.

\uabschnitt{control block:}
holds a set of registers (callee-saved registers, instruction pointer, stack
pointer) describing the execution context.

\uabschnitt{coroutine:}
enables explicit suspend and resume of its progress via additional operations by
preserving execution state and thus provides an enhanced control flow.\\
Coroutines have following characteristics\cite{N3985}:
\begin{itemize}
    \item values of local data persist between successive context switches
    \item execution is suspended as control leaves coroutine and resumed at
          certain time later
    \item symmetric or asymmetric control transfer-mechanism
    \item first-class object or compiler-internal structure
    \item stackless or stackfull
\end{itemize}

\uabschnitt{asymmetric coroutine:}
provides two distingt operations for contexts switch - one operation to
resume and one operation to suspend the coroutine.\\
A asymmetric coroutine is tightly coupled with its caller, e.g. on suspending
the coroutine transferes the execution control back to its invoker.\\
Usually used in the context of generators.

\uabschnitt{symmetric coroutine:}
only one operation to resume/suspend the context is available.\\
A symmetric coroutine does not know its caller, e.g. the execution control can
passed to any other symmetric coroutine.\\
Usually used to implement user-mode threads.

\uabschnitt{fiber/user-mode thread:}
execute tasks in an cooperative multitasking environment involving a
scheduler. Coroutines and fibers are distinct (see N4024\cite{N4024}).

\uabschnitt{resumeable function:}
N4134\cite{\N4134} describes resumable functions as an efficient language
supported mechanism for stackless coroutines introducing two new keywords -
\await and \yield. Resumable functions are equivalent to asymmetric coroutines.\\
Characteristics of resumkable functions:
\begin{itemize}
    \item stackless
    \item allocates memory (activation frame) for the body of the resumable
          function to store local data, registers etc.
    \item thight coupling between caller and resumable function (asymmetric
          control transfer-mechanism)
    \item implicit \textit{return}-statement\cite{N4134} (code-transformation)
          \cppf{N4134/fib.cpp}
\end{itemize}

\uabschnitt{linked stack:}
also known as \textit{split stack}\cite{gccsplit} or
\textit{segmented stack}\cite{llvmseg}, represents a stack with a non-contigous
address-range.\\
Applications compiled with support for linked stacks can use (link against)
libraries not supporting linked stacks (see GCC's documentation\cite{gccsplit},
chapter 'Backward compatibility').

\abschnitt{Introduction}
Traditionally C++ code is run on a linear stack, i.e. the activation
records are allocated in strict \emph{last-in-first-out} order. This stack
model allocates activation records on function call/return by
incrementing/decrementing the stack pointer.\\
But in the context of coroutines, that is, switching between different execution
contexts, a linear stack introduces problems. Calling a function creates an
activation record on the stack which is removed if the function returns.
But for a suspended coroutine the activation record {\bfseries must not} be
{\bfseries removed}!
\newline
Consider the following scenario:
\begin{itemize}
    \item Assume the processor stack is built in descending order: that is, a
          PUSH instruction decrements the stack pointer register. Call the
          stack pointer's initial value SP0.
    \item Function \main enters coroutine \cpp{C()}.
    \item \cpp{C()}'s prolog allocates a stack frame of
          size \cpp{sizeof(C::frame)} by decrementing SP. SP is now at SP1 =
          (SP0 - \cpp{sizeof(C::frame)}).
    \item \cpp{C()} suspends, returning control to \main. \main must find its
          stack frame at SP = SP0.
    \item \main now calls function \cpp{F()}.
    \item \cpp{F()}'s prolog allocates a stack frame of
          size \cpp{sizeof(F::frame)} by decrementing SP. SP is now at SP2 =
          (SP0 - \cpp{sizeof(F::frame)}).
    \item Unless either \cpp{(sizeof(C::frame) == 0)}
          or \cpp{(sizeof(F::frame) == 0)}, any data written by \cpp{F()} to
          its own stack frame will necessarily overwrite any data saved
          by \cpp{C()} in its stack frame.
    \item \cpp{F()} returns. SP is back to SP0.
    \item \main resumes \cpp{C()}. SP is set to SP1.
    \item \cpp{C()} attempts to access data in its stack frame -- which has
          been overwritten by \cpp{F()}. We are now in the realm of Undefined
          Behavior.
\end{itemize}

In order to prevent stack corruption, a stackless coroutine uses a heap-allocated
activation record (N4134\cite{N4134}), while stackful coroutines use a side
stack (N3985\cite{N3985}).\\
Since an N4134 stackless resumable function uses \emph{suspend by return},
when it suspends, the stack pointer is restored to its position before the
resumable function was called. While executing, a resumable function can
consume additional space in the linear processor stack; it can call
traditional functions. But they must all return before the resumable function
suspends. If resumable function \cpp{A()} calls function \cpp{B()}, and
\cpp{B()} wishes to suspend, \cpp{B()} must also be a resumable function. Thus
the term \emph{stackless}: a suspended resumable function leaves no activation
record on the linear processor stack. A single linear stack can be reused by
an arbitrary number of suspended resumable functions.\\
Using a side stack permits \emph{stackful} coroutines to use \emph{suspend by
call}. An arbitrary number of ordinary stack frames can be left on the side
stack for a suspended coroutine context; arbitrary stack frames can be pushed
or popped on the currently-active stack, independently of any suspended stack.\\
This is the fundamental difference between stackless and stackful coroutines.\\
\newline
Traditional stack management -- a single linear stack per thread -- is
inadequate for coroutines because coroutines must outlive the context in which
they were created.
\newpage

\abschnitt{Discussion}
N4397\cite{N4397} describes \ectx as a mechanism to implement stackful context
switching (for instance coroutines). Each context owns its own side stack.\\
How can this formalism be used to express \emph{stackless} as well as
\emph{stackful} execution contexts? The answer is the concept of
\emph{suspend-by-calling}.

\uabschnitt{Calling convention}
A calling convention is a scheme, part of the ABI\footnote{Application Binary
Interfaces; an executable must be conform to, in order to be executable in the
specified execution environment}, that describes how a subroutine must be
called. This includes parameter list, return address, \emph{stack layout} and
cleanup.\\
Some calling conventions (for instance x86 architecture) require that data like
\emph{parameter list} as well as \emph{return address} are stored on the caller's
stack before the subroutine is invoked. Other calling conventions of other
architectures (for instance ARM's AAPCS) do not have this constraint
\footnote{AAPCS64: parameters in registers R0-R7/V0-V7, return address in link
register LR}, e.g. stack consumption is minimized\footnote{of course a long
parameter list requires the stack; but this is negligible as shown in the text}
\newline
The proposed syntax for stackless and stackful context switching requires that
the stack is clean: no parameter list and no return address may remain on the
caller's stack when a context switch happens. This is required since a
coroutine can outlive the context in which it was created.\\
In other words, those parts that would remain on the caller's stack must be
preserved and restored by a context switch (member function \ectxop). This
supports the \emph{suspend-by-calling} concept. Previous proposals like
N4134\cite{N4134} describe a \emph{suspend-by-returning} mechanism, i.e. the
coroutine is suspended by calling \yield etc. (for resumable functions a return
value transformation happens -- the return value is substituted by a future-like
object).

\uabschnitt{Suspend-by-calling}
\emph{suspend-by-calling} requires a symmetric transfer of execution control
as well as first-class objects in order to specify the next context to be
resumed.\\
This enables an arbitrary flow of context switches providing a broad range of
control flows (for instance \emph{delimited continuations}).\\
As a consequence the part belonging to the called function on the caller's stack
must be popped and preserved in the caller's \emph{capture record}. The caller
context suspends by calling\\
\ectxop of another execution context. Other execution contexts are able to use
the stack in the meanwhile. If the suspended context is resumed, the preserved
data are pushed to the stack and execution returns from \ectxop.

\paragraph*{Prologue}
The prologue of \ectxop updates the capture record (CPU registers) and pops
some parts (part of callee's stack frame) from caller's stack and preserves the
data into caller's capture record too.

\paragraph*{Epilogue}
If the context (caller context from above) is resumed, the \emph{epilogue} of\\
\ectxop loads the capture record and pushes callee's partial stack frame on
caller's stack. The calling convention remains intact, and the code which
called\\
\ectxop can not distinguish between an ordinary function and a context switch.
It's completely transparent to the caller.

\paragraph*{Flow of control}
Because \ectx uses a symmetric execution control transfer mechanism, the flow
of control can be arbitrary.
\cppf{N4398/cycle}
As shown in the example, the contexts of \emph{ctx1}, \emph{ctx2} and
\emph{ctx3} form a cycle of flow of control.
\graphc{cycle}
The cycle is started by calling \cpp{ctx3()} from the main context
(\emph{$ctx_m$}, created on start-up). Context \emph{$ctx_3$} starts
\emph{$ctx_2$} while \emph{$ctx_2$} resumes \emph{$ctx_1$}. Context
\emph{$ctx_1$} is suspended by resuming \emph{$ctx_3$} with \cpp{(*other)()}.
Function \cpp{ctx2()} returns in \emph{$ctx_3$} and \cpp{ctx1()} next resumes
\emph{$ctx_1$}. Context \emph{$ctx_1$} terminates after returning from
\cpp{(*other)()}.\\
After termination of \emph{$ctx_1$}, \emph{$ctx_3$} is resumed because it has
become the parent context of \emph{$ctx_1$} (by calling \cpp{ctx1()}). As
\emph{$ctx_3$} terminates, the main context \emph{$ctx_m$} is resumed (return
from \cpp{ctx3()}).

\uabschnitt{Stackless and stackful}
The compiler allocates for a \emph{stackless context} only one, suitable sized,
capture record and for a \emph{stackful context} a side stack (linked stack,
e.g. non-contiguous, growing on demand).\\
In order to decide what kind of context has to be generated, the compiler has to
analyse the toplevel context function.The following use cases must be
distinguished:
\begin{itemize}
    \item no context switch: generate an ordinary function
    \item context switch at toplevel: create a \emph{stackless} context
    \item in all other cases: generate a \emph{stackful} context
\end{itemize}
\cppf{N4398/fibonacci}
In other words, if \ectxop is called inside the body of the toplevel context
function (as shown in Fibonacci example above) a \emph{stackless} context is
sufficient.
\cppf{N4398/parser}
Calling a context switch from a nested call stack requires a \emph{stackful}
context.

\abschnitt{Design}
Class \ectx is derived from the work on boost.context\cite{bcontext} - it
provides a small, basic API, suitable to implement high-level APIs for stackful
coroutines (N3985\cite{N3985}, boost.coroutine2\cite{bcoroutine2}) and user-mode
threads (executing tasks in a
cooperative multitasking environment, boost.fiber\cite{bfiber}).\\

\uabschnitt{Class \ectx}
The interface contains only one operations to switch the execution context
\ectxop (symmetric operation) - that implies that \ectx needs to specify
explictly to which other context the execution control has to be transferred to.
Exchanging data between execution context's requires the use of lambda captures.

\paragraph*{member functions}
\subparagraph*{(constructor)}
constructs new execution context\\

\begin{tabular}{ l l }
    \midrule

    \cpp{execution_context(StackAlloc salloc,Fn&& fn,Args&&... args);} & (1)\\

    \midrule

    \cpp{execution_context(execution_context const& other)=default;} & (2)\\

    \midrule

    \cpp{execution_context(execution_context&& other)=default;} & (3)\\

    \midrule
\end{tabular}

\begin{description}
    \item[1)] creates a \ectx
              \begin{itemize}
                  \item \textit{salloc} allocates/deallocates stack
                  \item \textit{fn} function executed in the new context
                  \item \textit{args} parameter pack passed to \textit{fn}
              \end{itemize}
    \item[2)] copies \ectx, e.g. underlying control block is shared
    \item[3)] moves underlying control block to new \ectx
\end{description}

{\bf Notes}
\newline
If an instance of \ectx is copied, both instances share the same underlying
control block (CPU registers, stack). Resuming one instance modifies the
control block (internal state) of the other \ectx too.\\
If this is behaviour is not permitted, the stack has to be copied. That requires
identification and modification of local variables pointing to address of the
stack.\\

\subparagraph*{(destructor)}
destroys a execution context\\

\begin{tabular}{ l l }
    \midrule

    \cpp{\~execution_context();} & (1)\\

    \midrule
\end{tabular}

\begin{description}
    \item[1)] destroys a \ectx. If associated with a context of execution and
              holds the last reference to the internal control block, then the
              context of execution is destroyed too. Specifically, the stack is
              unwound.\\
\end{description}

\subparagraph*{operator=}
moves the coroutine object\\

\begin{tabular}{ l l }
    \midrule

    \cpp{execution_context & operator=(execution_context&& other);} & (1)\\

    \midrule

    \cpp{execution_context & operator=(const execution_context& other);} & (2)\\

    \midrule
\end{tabular}

\begin{description}
    \item[1)] assigns the state of \textit{other} to *this using move semantics
    \item[2)] copies the state of \textit{other} to *this, state (control block)
              is shared
\end{description}

{\bf Parameters}
\begin{description}
    \item[other]   another execution context to assign to this object\\
\end{description}

{\bf Return value}
\begin{description}
    \item[*this]
\end{description}

\subparagraph*{operator()}
jump context of execution\\

\begin{tabular}{ l l }
    \midrule

    \cpp{void operator()() noexcept;} & (1)\\

    \midrule
\end{tabular}

\begin{description}
    \item[1)] resumes the execution context\\
\end{description}

{\bf Exceptions}
\begin{description}
    \item[1)] noexcept specification: \cpp{noexcept}\\
\end{description}

{\bf Notes}
\newline
If an exception leaves this function \cpp{std::terminate()} is called.\\ 
If this function returns, \cpp{std::exit(0)} is called.

\subparagraph*{current}
accesses the current active execution context\\

\begin{tabular}{ l l }
    \midrule

    \cpp{static execution_context current();} & (1)\\

    \midrule
\end{tabular}

\begin{description}
    \item[1)] construct a instance of \ectx associated with the current active
              execution context\\
\end{description}

{\bf Notes}
\newline
The current active execution context is thread-specific.

\uabschnitt{Stackful resumable lambda}
Based on the implementation experience with \cpp{execution_context} in
boost.coroutine2\cite{bcoroutine2} and boost.fiber\cite{bfiber} the author
encountered that \cpp{execution_context} is almost always used together with
lambdas (passed as argument to the constructor of \ectx). Especially the
lambda captures are suitable to transport data between different execution
context's.\\
Why not construct \ectx directly with an 'resumable lambda'-like syntax?
\cppf{rl1.cpp}
The differences to N4244 are the absence of keyword \yield because of
symmetric context switching (== only one operation transfers the control of
execution), e.g. the target of a context switch must be explicitly specified.\\
Stackful resumable lambdas require keyword \resumable together with an hint
(attribute) about the type and size of the stack.\\
Hints are \cpp{fixedsize()} and \cpp{segmented()}:
\begin{itemize}
    \item \cpp{fixedsize(x)}: instructs the compiler to allocate an new (or
          re-use a cached) stack with a size of \textit{x} bytes
    \item \cpp{segmented(y)}: use a linked stack that grows on demand
          with an initial size of \textit{y} bytes
\end{itemize}
The example of an recursive descent parser using stackful resumable lambda would
look like:
\cppf{lparser.cpp}

\uabschnitt{A combined syntax for stackless and stackful context switching}
A syntax, combining stackless and stackful resumable lambdas, could be possible
too. Stackful resumable lambdas are identified by the additional stack hint
applied to lambda attribute \textit{resumable}. The absence of the hint would
tell the compiler to  create a stackless context.\\

\begin{tabular}{ l l }
    \midrule

    \cpp{[capture-list] (params) mutable resumable(hint) exceptions attribute -> ret \{body\}} & (1)\\

    \midrule
\end{tabular}

\begin{description}
    \item[1)] Full declaration
\end{description}

{\bf Parameters}
\begin{description}
    \item[mutable]      allows to modify parameters captured by copy
    \item[resumable]    identify resumable operation
    \item[hint]         stack type hint:
                        \begin{itemize}
                            \item <no hint specified>: create stackless
                                  resumable lambda
                            \item \textit{fixedsize(x)}: create stackful
                                  resumable lambda; fixed size stack
                            \item \textit{segmented(x)}: create stackful
                                  resumable lambda; stack grows on demand
                        \end{itemize}
    \item[exceptions]   only \textit{noexcept} allowed; no exception is
                        permitted to leave the body otherwise
                        \cpp{std::terminate()} is called
    \item[attribute]    attributes for \cpp{operator()}
    \item[capture-list] list of captures
    \item[params]       only empty parameter-list allowed
    \item[ret]          only \textit{void} allowed; resumable lambda returns nothing
    \item[body]         function body\\
\end{description}

If the compiler can analyse the function body of the resumable lambda, then the
compiler is free ton transform the stackful into a stackless resumable lambda as
an optimization.

\abschnitt{Acknowledgement}
I'd like to thank Nat Goodspeed for his support.


%//////////////////////////////////////////////////////////////////////////////

\newpage
\addcontentsline{toc}{subsection}{References}
\begin{thebibliography}{99}

    \bibitem{N3985}
        \href{http://www.open-std.org/jtc1/sc22/wg21/docs/papers/2014/n3985.pdf}
        {N3985: A proposal to add coroutines to the C++ standard library, Revision 1}

    \bibitem{N4024}
        \href{http://www.open-std.org/jtc1/sc22/wg21/docs/papers/2014/n4024.pdf}
        {N4024: Distinguishing coroutines and fibers}

    \bibitem{N4134}
        \href{http://www.open-std.org/jtc1/sc22/wg21/docs/papers/2014/n4134.pdf}
        {N4134: Resumable Functions v.2}

    \bibitem{N4244}
        \href{http://www.open-std.org/jtc1/sc22/wg21/docs/papers/2014/n4244.pdf}
        {N4244: Resumable Lambdas}

    \bibitem{N4398}
        \href{http://www.open-std.org/jtc1/sc22/wg21/docs/papers/2015/n4398.pdf}
        {N4398: A unified syntax for stackless and stackful coroutines}

    \bibitem{gccsplit}
        \href{http://gcc.gnu.org/wiki/SplitStacks}
        {Split Stacks / GCC}

    \bibitem{llvmseg}
        \href{http://llvm.org/releases/3.0/docs/SegmentedStacks.html}
        {Segmented Stacks / LLVM}

    \bibitem{bcontext}
        Library \emph{Boost.Context}:
        \href{https://github.com/boostorg/context} {git repo},
        \href{http://www.boost.org/doc/libs/1_58_0/libs/context/doc/html/index.html} {documentation}

    \bibitem{bcoroutine2}
        Library \emph{Boost.Coroutine2}:
        \href{https://github.com/boostorg/coroutine2} {git repo},
        \href{http://olk.github.io/libs/coroutine2/doc/html/index.html} {documentation}

    \bibitem{bfiber}
        Library \emph{Boost.Fiber}:
        \href{https://github.com/olk/boost-fiber} {git repo},
        \href{http://olk.github.io/libs/fiber/doc/html/index.html} {documentation}

    \bibitem{cactusstack}
        \emph{cactus stack}:
        \href{http://c2.com/cgi/wiki?CactusStack} {cactus stack},
        \href{http://en.wikipedia.org/wiki/Parent_pointer_tree} {parent pointer tree}

\end{thebibliography}


%//////////////////////////////////////////////////////////////////////////////

\abschnitt{A. x86\_64 SYSV calling convention}

\asmf{foo.S}

\asmf{bar.S}


%//////////////////////////////////////////////////////////////////////////////

\end{document}
