\pdfoutput=1
\pdfcompresslevel=9
\pdfinfo
{
	/Title (A proposal to add coroutines to the C++ standard library)
	/Author (Oliver Kowalke, Nat Goodspeed)
	/Keywords (C++ coroutine )
}

%//////////////////////////////////////////////////////////////////////////////

\documentclass[a4paper,10pt,DIV15]{scrartcl}

\usepackage{overcite}
\usepackage[british]{babel}
\usepackage[latin1]{inputenc}
\usepackage[T1]{fontenc}
\usepackage{booktabs}
\usepackage{fancyhdr}
\usepackage{listings}
\usepackage{rotating}
\usepackage{xcolor}
\usepackage{xspace}
\usepackage[colorlinks=true,urlcolor=blue]{hyperref}

%//////////////////////////////////////////////////////////////////////////////

\setlength{\parindent}{0pt} 

\makeatletter
    \renewcommand*\l@subsection{\@dottedtocline{2}{0em}{2.3em}}
    \renewcommand*\l@subsection{\@dottedtocline{3}{0em}{3.2em}}
    \renewcommand{\tableofcontents}{%
        \@starttoc{toc}
    }
\makeatother

%\renewcommand{\thesubsection}{\Roman{subsection}}

\newcommand{\pdfimg}[1]{\pdfximage{pics/#1}\pdfrefximage\pdflastximage}
\newcommand{\img}[1]{\mbox{\pdfimg{#1}}}
\newcommand{\imgc}[1]{\begin{center}\img{#1}\end{center}}

\newcommand{\cpp}[1]{
    \lstinline[
        language=C++,
        basicstyle=\ttfamily\color{black},
        keywordstyle=\color{blue},
        commentstyle=\color{green},
        stringstyle=\color{red}
    ] {#1}
}
\newcommand{\cppf}[1]{
    \lstinputlisting[
        language=C++,
        basicstyle=\ttfamily\color{black},
        keywordstyle=\color{blue},
        commentstyle=\color{red},
        stringstyle=\color{green}
    ] {code/#1}
}

\newcommand{\corobool}{\cpp{std::coroutine<>::operator safe_bool()}\xspace}
\newcommand{\coro}{\cpp{std::coroutine<>}\xspace}
\newcommand{\coroget}{\cpp{std::coroutine<>::get()}\xspace}
\newcommand{\coroiterator}{\cpp{std::coroutine<>::iterator}\xspace}
\newcommand{\coroop}{\cpp{std::coroutine<>::operator()(Args...)}\xspace}
\newcommand{\pullcorobool}{\cpp{std::pull_coroutine<>::operator safe_bool()}\xspace}
\newcommand{\pullcoro}{\cpp{std::pull_coroutine<>}\xspace}
\newcommand{\pullcoroget}{\cpp{std::pull_coroutine<>::get()}\xspace}
\newcommand{\pullcoroop}{\cpp{std::pull_coroutine<>::operator()()}\xspace}
\newcommand{\pushcoro}{\cpp{std::push_coroutine<>}\xspace}
\newcommand{\pushcoroop}{\cpp{std::push_coroutine<>::operator()(Arg)}\xspace}
\newcommand{\tuple}{\cpp{std::tuple<>}\xspace}

\newcommand{\await}{\textit{await}\xspace}
\newcommand{\csharp}{\textit{C#}\xspace}
\newcommand{\cblock}{\textit{control-block}\xspace}
\newcommand{\continuation}{\textit{continuation}\xspace}
\newcommand{\coopmultitasking}{\textit{cooperative multitasking}\xspace}
\newcommand{\corofunction}{\textit{coroutine-function}\xspace}
\newcommand{\escre}{\textit{escape-and-reenter}\xspace}
\newcommand{\escreops}{\textit{\escre~operations}\xspace}
\newcommand{\escreccomps}{\textit{\escre~recursive computations}\xspace}
\newcommand{\escrecloops}{\textit{\escre~loops}\xspace}
\newcommand{\resumfn}{\textit{resumeable function}\xspace}
\newcommand{\asyncops}{\textit{asynchronous-operations}\xspace}
\newcommand{\asyncres}{\textit{asynchronous-result}\xspace}

\newcommand{\awaitemu}{await_emu\cite{awaite}\xspace}
\newcommand{\boostasio}{boost.asio\cite{asio154}\xspace}
\newcommand{\boostcontext}{boost.context\cite{context154}\xspace}
\newcommand{\boostcoroutine}{boost.coroutine\cite{coroutine154}\xspace}
\newcommand{\boostcorosum}{boost.coroutine (Google Summer of Code 2006)\cite{coroutineSum2006}\xspace}

\newcommand{\abschnitt}[1]{
    \addcontentsline{toc}{subsection}{#1}
    \subsection*{#1}
}


%//////////////////////////////////////////////////////////////////////////////

\begin{document}

\small
\begin{tabbing}
    Document number: \=  \\
    Date:            \> 2014-04-21 \\
    Project:         \> Programming Language C++, Library Evolution Group\\
    Reply-to:        \> Oliver Kowalke (oliver dot kowalke at gmail dot com)\\
                     \> Nat Goodspeed ( nat at lindenlab dot com)\\
\end{tabbing}

\section*{A proposal to add coroutines to the C++ standard library (Revision 1)}

%//////////////////////////////////////////////////////////////////////////////

\tableofcontents

%//////////////////////////////////////////////////////////////////////////////

\paragraph*{Changes in this revision}
This document supersedes N3708. A new kind of coroutines - \scoro - is introduced
and additional examples (like recursive SAX parsing) are added.\\
A section explains the benfits of using coroutines in the context of event-based
asynchronous model.

\abschnitt{Introduction}
Traditionally C++ code is run on a linear stack, i.e. the activation
records are allocated in strict \emph{last-in-first-out} order. This stack
model allocates activation records on function call/return by
incrementing/decrementing the stack pointer.\\
But in the context of coroutines, that is, switching between different execution
contexts, a linear stack introduces problems. Calling a function creates an
activation record on the stack which is removed if the function returns.
But for a suspended coroutine the activation record {\bfseries must not} be
{\bfseries removed}!
\newline
Consider the following scenario:
\begin{itemize}
    \item Assume the processor stack is built in descending order: that is, a
          PUSH instruction decrements the stack pointer register. Call the
          stack pointer's initial value SP0.
    \item Function \main enters coroutine \cpp{C()}.
    \item \cpp{C()}'s prolog allocates a stack frame of
          size \cpp{sizeof(C::frame)} by decrementing SP. SP is now at SP1 =
          (SP0 - \cpp{sizeof(C::frame)}).
    \item \cpp{C()} suspends, returning control to \main. \main must find its
          stack frame at SP = SP0.
    \item \main now calls function \cpp{F()}.
    \item \cpp{F()}'s prolog allocates a stack frame of
          size \cpp{sizeof(F::frame)} by decrementing SP. SP is now at SP2 =
          (SP0 - \cpp{sizeof(F::frame)}).
    \item Unless either \cpp{(sizeof(C::frame) == 0)}
          or \cpp{(sizeof(F::frame) == 0)}, any data written by \cpp{F()} to
          its own stack frame will necessarily overwrite any data saved
          by \cpp{C()} in its stack frame.
    \item \cpp{F()} returns. SP is back to SP0.
    \item \main resumes \cpp{C()}. SP is set to SP1.
    \item \cpp{C()} attempts to access data in its stack frame -- which has
          been overwritten by \cpp{F()}. We are now in the realm of Undefined
          Behavior.
\end{itemize}

In order to prevent stack corruption, a stackless coroutine uses a heap-allocated
activation record (N4134\cite{N4134}), while stackful coroutines use a side
stack (N3985\cite{N3985}).\\
Since an N4134 stackless resumable function uses \emph{suspend by return},
when it suspends, the stack pointer is restored to its position before the
resumable function was called. While executing, a resumable function can
consume additional space in the linear processor stack; it can call
traditional functions. But they must all return before the resumable function
suspends. If resumable function \cpp{A()} calls function \cpp{B()}, and
\cpp{B()} wishes to suspend, \cpp{B()} must also be a resumable function. Thus
the term \emph{stackless}: a suspended resumable function leaves no activation
record on the linear processor stack. A single linear stack can be reused by
an arbitrary number of suspended resumable functions.\\
Using a side stack permits \emph{stackful} coroutines to use \emph{suspend by
call}. An arbitrary number of ordinary stack frames can be left on the side
stack for a suspended coroutine context; arbitrary stack frames can be pushed
or popped on the currently-active stack, independently of any suspended stack.\\
This is the fundamental difference between stackless and stackful coroutines.\\
\newline
Traditional stack management -- a single linear stack per thread -- is
inadequate for coroutines because coroutines must outlive the context in which
they were created.
\newpage

\abschnitt{Motivation}

This proposal refers to \boostcoroutine as reference implementation - providing
a test suite and examples (some are described in this section).\\
\newline
In order to support a broad range of execution control behaviour \pushcoro and\\
\pullcoro can be used to \escrecloops, to \escreccomps~and for \coopmultitasking
helping to solve problems in a much simpler and more elegant way than with only
a single flow of control.

\subsubsection*{'same fringe' problem}
The advantages can be seen particularly clearly with the use of a recursive
function, such as traversal of trees.\\
If traversing two different trees in the same deterministic order produces the
same list of leaf nodes, then both trees have the same fringe even if the tree
structure is different.\\
\newline
The same fringe problem could be solved using coroutines by iterating over the
leaf nodes and comparing this sequence via \cpp{std::equal()}. The range of data
values is generated by function \cpp{traverse()} which recursively traverses the
tree and passes each node's data value to its \pushcoro.\\
\pushcoro suspends the recursive computation and transfers the data value to
the main execution context.\\
\pullcoroiterator, created from \pullcoro, steps over those data values and
delivers them to \cpp{std::equal()} for comparison. Each increment of \pullcoroiterator
resumes \cpp{traverse()}. Upon return from \cpp{iterator::operator++()}, either
a new data value is available, or tree traversal is finished (iterator is
invalidated).
\cppf{same_fringe.cpp}

\subsubsection*{asynchronous operations with boost.asio}
In the past the code using asio's \asyncops was scattered by callbacks.
\boostasio provides with its new \asyncres feature a new way to simplify the
code and make it easier to read.\\
Proposal 'N3747: A Universal Model for Asynchronous Operations'\cite{n3747}
describes the usage of coroutines in the context of asynchronous operations.\\
\yieldcontext uses internally \boostcoroutine:
\cppf{asyncres.cpp} 

\subsubsection*{\csharp \await}
\csharp contains the two keywords \async and \await. \async introduces a
control flow that involves awaiting asynchronous operations. The compiler
reorganizes the code into a continuation-passing style. \await wraps the rest
of the function after calling \await into a continuation if the asynchronous
operation has not yet completed.\\
The project \awaitemu uses \boostcoroutine for a proof-of-concept
demonstrating the implementation of a full emulation of \csharp \await as a
library extension. Because of stackful coroutines \await is \textbf{not limited}
by "one level" as in \csharp.\\
Evgeny Panasyuk describes the advantages of \boostcoroutine over \await at
\channelnine.
\cppf{await.cpp}

\abschnitt{Impact on the Standard}

This proposal is a library extension. It does not require changes to any
standard classes, functions or headers. It can be implemented in C++03 and C++11
and requires no core language changes.

\abschnitt{Design}
Class \ectx is derived from the work on boost.context\cite{bcontext} - it
provides a small, basic API, suitable to implement high-level APIs for stackful
coroutines (N3985\cite{N3985}, boost.coroutine2\cite{bcoroutine2}) and user-mode
threads (executing tasks in a
cooperative multitasking environment, boost.fiber\cite{bfiber}).\\

\uabschnitt{Class \ectx}
The interface contains only one operations to switch the execution context
\ectxop (symmetric operation) - that implies that \ectx needs to specify
explictly to which other context the execution control has to be transferred to.
Exchanging data between execution context's requires the use of lambda captures.

\paragraph*{member functions}
\subparagraph*{(constructor)}
constructs new execution context\\

\begin{tabular}{ l l }
    \midrule

    \cpp{execution_context(StackAlloc salloc,Fn&& fn,Args&&... args);} & (1)\\

    \midrule

    \cpp{execution_context(execution_context const& other)=default;} & (2)\\

    \midrule

    \cpp{execution_context(execution_context&& other)=default;} & (3)\\

    \midrule
\end{tabular}

\begin{description}
    \item[1)] creates a \ectx
              \begin{itemize}
                  \item \textit{salloc} allocates/deallocates stack
                  \item \textit{fn} function executed in the new context
                  \item \textit{args} parameter pack passed to \textit{fn}
              \end{itemize}
    \item[2)] copies \ectx, e.g. underlying control block is shared
    \item[3)] moves underlying control block to new \ectx
\end{description}

{\bf Notes}
\newline
If an instance of \ectx is copied, both instances share the same underlying
control block (CPU registers, stack). Resuming one instance modifies the
control block (internal state) of the other \ectx too.\\
If this is behaviour is not permitted, the stack has to be copied. That requires
identification and modification of local variables pointing to address of the
stack.\\

\subparagraph*{(destructor)}
destroys a execution context\\

\begin{tabular}{ l l }
    \midrule

    \cpp{\~execution_context();} & (1)\\

    \midrule
\end{tabular}

\begin{description}
    \item[1)] destroys a \ectx. If associated with a context of execution and
              holds the last reference to the internal control block, then the
              context of execution is destroyed too. Specifically, the stack is
              unwound.\\
\end{description}

\subparagraph*{operator=}
moves the coroutine object\\

\begin{tabular}{ l l }
    \midrule

    \cpp{execution_context & operator=(execution_context&& other);} & (1)\\

    \midrule

    \cpp{execution_context & operator=(const execution_context& other);} & (2)\\

    \midrule
\end{tabular}

\begin{description}
    \item[1)] assigns the state of \textit{other} to *this using move semantics
    \item[2)] copies the state of \textit{other} to *this, state (control block)
              is shared
\end{description}

{\bf Parameters}
\begin{description}
    \item[other]   another execution context to assign to this object\\
\end{description}

{\bf Return value}
\begin{description}
    \item[*this]
\end{description}

\subparagraph*{operator()}
jump context of execution\\

\begin{tabular}{ l l }
    \midrule

    \cpp{void operator()() noexcept;} & (1)\\

    \midrule
\end{tabular}

\begin{description}
    \item[1)] resumes the execution context\\
\end{description}

{\bf Exceptions}
\begin{description}
    \item[1)] noexcept specification: \cpp{noexcept}\\
\end{description}

{\bf Notes}
\newline
If an exception leaves this function \cpp{std::terminate()} is called.\\ 
If this function returns, \cpp{std::exit(0)} is called.

\subparagraph*{current}
accesses the current active execution context\\

\begin{tabular}{ l l }
    \midrule

    \cpp{static execution_context current();} & (1)\\

    \midrule
\end{tabular}

\begin{description}
    \item[1)] construct a instance of \ectx associated with the current active
              execution context\\
\end{description}

{\bf Notes}
\newline
The current active execution context is thread-specific.

\uabschnitt{Stackful resumable lambda}
Based on the implementation experience with \cpp{execution_context} in
boost.coroutine2\cite{bcoroutine2} and boost.fiber\cite{bfiber} the author
encountered that \cpp{execution_context} is almost always used together with
lambdas (passed as argument to the constructor of \ectx). Especially the
lambda captures are suitable to transport data between different execution
context's.\\
Why not construct \ectx directly with an 'resumable lambda'-like syntax?
\cppf{rl1.cpp}
The differences to N4244 are the absence of keyword \yield because of
symmetric context switching (== only one operation transfers the control of
execution), e.g. the target of a context switch must be explicitly specified.\\
Stackful resumable lambdas require keyword \resumable together with an hint
(attribute) about the type and size of the stack.\\
Hints are \cpp{fixedsize()} and \cpp{segmented()}:
\begin{itemize}
    \item \cpp{fixedsize(x)}: instructs the compiler to allocate an new (or
          re-use a cached) stack with a size of \textit{x} bytes
    \item \cpp{segmented(y)}: use a linked stack that grows on demand
          with an initial size of \textit{y} bytes
\end{itemize}
The example of an recursive descent parser using stackful resumable lambda would
look like:
\cppf{lparser.cpp}

\uabschnitt{A combined syntax for stackless and stackful context switching}
A syntax, combining stackless and stackful resumable lambdas, could be possible
too. Stackful resumable lambdas are identified by the additional stack hint
applied to lambda attribute \textit{resumable}. The absence of the hint would
tell the compiler to  create a stackless context.\\

\begin{tabular}{ l l }
    \midrule

    \cpp{[capture-list] (params) mutable resumable(hint) exceptions attribute -> ret \{body\}} & (1)\\

    \midrule
\end{tabular}

\begin{description}
    \item[1)] Full declaration
\end{description}

{\bf Parameters}
\begin{description}
    \item[mutable]      allows to modify parameters captured by copy
    \item[resumable]    identify resumable operation
    \item[hint]         stack type hint:
                        \begin{itemize}
                            \item <no hint specified>: create stackless
                                  resumable lambda
                            \item \textit{fixedsize(x)}: create stackful
                                  resumable lambda; fixed size stack
                            \item \textit{segmented(x)}: create stackful
                                  resumable lambda; stack grows on demand
                        \end{itemize}
    \item[exceptions]   only \textit{noexcept} allowed; no exception is
                        permitted to leave the body otherwise
                        \cpp{std::terminate()} is called
    \item[attribute]    attributes for \cpp{operator()}
    \item[capture-list] list of captures
    \item[params]       only empty parameter-list allowed
    \item[ret]          only \textit{void} allowed; resumable lambda returns nothing
    \item[body]         function body\\
\end{description}

If the compiler can analyse the function body of the resumable lambda, then the
compiler is free ton transform the stackful into a stackless resumable lambda as
an optimization.

\abschnitt{Technical Specification}

\subsubsection*{std::coroutine<>::pull\_type}
Defined in header \cpp{<coroutine>}.\\
\begin{tabular}{ l }
    \midrule

    \cpp{template<class T> class coroutine<T>::pull_type;}\\

    \midrule

    \cpp{template<class T> class coroutine<T&>::pull_type;}\\

    \midrule

    \cpp{template<> class coroutine<void>::pull_type;}\\

    \midrule
\end{tabular}
\newline
The class \pullcoro provides a mechanism to receive data values from
another execution context.\\

\paragraph*{member types\\}
\begin{tabular}{ l l l }
    \midrule

    \cpp{iterator} & std::input\_iterator & (not defined for coroutine<void>::pull\_type template specialization)\\

    \midrule
\end{tabular}

\paragraph*{member functions}
\subparagraph*{(constructor)}
constructs new coroutine\\

\begin{tabular}{ l l }
    \midrule

    \cpp{pull_type();} & (1)\\

    \midrule

    \cpp{pull_type(Function&& fn);} & (2)\\

    \midrule

    \cpp{pull_type(pull_type&& other);} & (3)\\

    \midrule

    \cpp{pull_type(const pull_type& other)=delete;} & (4)\\

    \midrule
\end{tabular}

\begin{description}
    \item[1)] creates a \pullcoro which does not represent a context of execution
    \item[2)] creates a \pullcoro object and associates it with a execution
              context
    \item[3)] move constructor, constructs a \pullcoro object to represent a
              context of execution that was represented by \textit{other}, after this
              call \textit{other} no longer represents a coroutine
    \item[4)] copy constructor is deleted; coroutines are not copyable\\
\end{description}

{\bf Notes}
\newline
Return values from the \corofunction are accessible via \pullcoroget.\\
If the \corofunction throws an exception, this exception is re-thrown when the
caller returns from\\
\pullcoroop.\\

{\bf Parameters}
\begin{description}
    \item[other]  another coroutine object with which to construct this coroutine object
    \item[fn]     function to execute in the new coroutine\\
\end{description}

{\bf Exceptions}
\begin{description}
    \item[1), 3)] noexcept specification: \cpp{noexcept}
    \item[2)] \cpp{std::system_error} if the coroutine could not be started
                  - the exception may represent a implementation-specific error
                  condition; re-throw user defined exceptions from \corofunction\\
\end{description}

{\bf Example}
\cppf{fibonacci.cpp}

\subparagraph*{(destructor)}
destructs a coroutine\\

\begin{tabular}{ l l }
    \midrule

    \cpp{\~pull_type();} & (1)\\

    \midrule
\end{tabular}

\begin{description}
    \item[1)] destroys a \pullcoro. If that \pullcoro is associated with a context of execution,
              then the context of execution is destroyed too. Specifically,
              its stack is unwound.\\
\end{description}

\subparagraph*{operator=}
moves the coroutine object\\

\begin{tabular}{ l l }
    \midrule

    \cpp{pull_type & operator=(pull_type&& other);} & (1)\\

    \midrule

    \cpp{pull_type & operator=(const pull_type& other)=delete;} & (2)\\

    \midrule
\end{tabular}

\begin{description}
    \item[1)] assigns the state of \textit{other} to *this using move semantics
    \item[2)] copy assignment is deleted; coroutines are not copyable\\
\end{description}

{\bf Parameters}
\begin{description}
    \item[other]   another coroutine object to assign to this coroutine object\\
\end{description}

{\bf Return value}
\begin{description}
    \item[*this]
\end{description}

{\bf Exceptions}
\begin{description}
    \item[1)] noexcept specification: \cpp{noexcept}\\
\end{description}

\subparagraph*{operator bool}
indicates whether context of execution is still valid and a return value can be
retrieved, or \corofunction has finished\\

\begin{tabular}{ l l }
    \midrule

    \cpp{operator bool();} & (1)\\

    \midrule
\end{tabular}

\begin{description}
    \item[1)] evaluates to true if coroutine is not complete (\corofunction has
        not terminated)\\
\end{description}

{\bf Exceptions}
\begin{description}
    \item[1)] noexcept specification: \cpp{noexcept}\\
\end{description}

\subparagraph*{operator()}
jump context of execution\\

\begin{tabular}{ l l }
    \midrule

    \cpp{pull_type & operator()();} & (1)\\

    \midrule
\end{tabular}

\begin{description}
    \item[1)] transfer control of execution to \corofunction\\
\end{description}

{\bf Notes}
\newline
It is important that the coroutine is still valid (\cpp{operator bool()}
returns \cpp{true}) before calling this function, otherwise it results in
undefined behaviour.\\

{\bf Return value}
\begin{description}
    \item[*this]
\end{description}

{\bf Exceptions}
\begin{description}
    \item[1)] \cpp{std::system_error} if control of execution could not be
              transferred to other execution context - the exception may
              represent a implementation-specific error condition; re-throw
              user-defined exceptions from \corofunction\\
\end{description}

\subparagraph*{get}
accesses the current value from \corofunction\\

\begin{tabular}{ l l l }
    \midrule

    \cpp{R get();} & (1) & (member of generic template)\\

    \midrule

    \cpp{R& get();} & (2) & (member of generic template)\\

    \midrule

    \cpp{void get()=delete;} & (3) & (only for coroutine<void>::pull\_type template specialization)\\

    \midrule
\end{tabular}

\begin{description}
    \item[1)] access values returned from \corofunction (if move-assignable, the
              value is moved, otherwise copied)
    \item[2)] access reference returned from \corofunction\\
\end{description}

{\bf Notes}
\newline
It is important that the coroutine is still valid (\cpp{operator bool()}
returns \cpp{true}) before calling this function, otherwise it results in
undefined behaviour.\\
If type T is move-assignable, it will be returned using move semantics. With
such a type, if you call \get a second time before calling
\cpp{operator()()}, \get will throw an exception -- see below.\\

{\bf Return value}
\begin{description}
    \item[R] return type is defined by coroutine's template argument
    \item[void] coroutine does not support \get\\
\end{description}

{\bf Exceptions}
\begin{description}
    \item[1)] Once a particular move-assignable value has already been
        retrieved by \get, any subsequent \get call throws
        \cpp{std::coroutine_error} with an error-code
        \cpp{std::coroutine_errc::no_data} until \cpp{operator()()} is called.\\
\end{description}

\subparagraph*{swap}
swaps two coroutine objects\\

\begin{tabular}{ l l }
    \midrule

    \cpp{void swap(pull_type& other);} & (1)\\

    \midrule
\end{tabular}

\begin{description}
    \item[1)] exchanges the underlying context of execution of two coroutine
              objects\\
\end{description}

{\bf Exceptions}
\begin{description}
    \item[1)] noexcept specification: \cpp{noexcept}\\
\end{description}

\paragraph*{non-member functions}
\subparagraph*{std::swap}
Specializes \cpp{std::swap} for \pullcoro and swaps the underlying context of
lhs and rhs.\\

\begin{tabular}{ l l }
    \midrule

    \cpp{void swap(pull_type& lhs,pull_type& rhs);} & (1)\\

    \midrule
\end{tabular}

\begin{description}
    \item[1)] exchanges the underlying context of execution of two coroutine
              objects by calling \cpp{lhs.swap(rhs)}.\\
\end{description}

{\bf Exceptions}
\begin{description}
    \item[1)] noexcept specification: \cpp{noexcept}\\
\end{description}

\subparagraph*{std::begin}
Specializes \cpp{std::begin} for \pullcoro.\\

\begin{tabular}{ l l }
    \midrule

    \cpp{template<class R> coroutine<R>::pull_type::iterator begin(coroutine<R>::pull_type& c);} & (1)\\

    \midrule
\end{tabular}

\begin{description}
    \item[1)] creates and returns a \cpp{std::input_iterator}\\
\end{description}

\subparagraph*{std::end}
Specializes \cpp{std::end} for \pullcoro.\\

\begin{tabular}{ l l }
    \midrule

    \cpp{template<class R> coroutine<R>::pull_type::iterator end(coroutine<R>::pull_type& c);} & (1)\\

    \midrule
\end{tabular}

\begin{description}
    \item[1)] creates and returns a \cpp{std::input_iterator} indicating the termination of the \corofunction\\
\end{description}

Incrementing the iterator switches the execution context.
\newline
When a main-context calls \cpp{iterator::operator++()} on an iterator obtained
from an explicitly-instantiated\\
\pullcoro, it must compare the incremented
value with the iterator returned by \cpp{std::end()}. If they are unequal, the
\corofunction has passed a new data value, which can be accessed via
\cpp{iterator::operator*()}. Otherwise the \corofunction has terminated and the
incremented iterator has become invalid.\\
When a \pushcoro's \corofunction calls \cpp{iterator::operator++()} on an iterator
obtained from the \pullcoro passed by the library, control is transferred back
to the main-context. The main-context may never pass another data value. From
the \corofunction's point of view, the \cpp{iterator::operator++()} call may
never return. If it does return, the main-context has passed a new data value,
which can be accessed via \cpp{iterator::operator*()}.\\
A function written to compare the incremented iterator with the iterator
returned by \cpp{std::end()} can be used in either situation.\\
If the return-type is move-assignable the first call to \cpp{iterator::operator*()}
moves the value. After that, any subsequent call to \cpp{iterator::operator*()} throws an
exception (\cpp{std::coroutine_error}) until \cpp{iterator::operator++()} is called.\\
The iterator is forward-only.\\

{\bf Example}
\cppf{input_iterator.cpp}


\subsubsection*{std::coroutine<>::push\_type}
Defined in header \cpp{<coroutine>}.\\
\begin{tabular}{ l }
    \midrule

    \cpp{template<class T> class coroutine<T>::push_type;}\\

    \midrule

    \cpp{template<class T> class coroutine<T&>::push_type;}\\

    \midrule

    \cpp{template<> class coroutine<void>::push_type;}\\

    \midrule
\end{tabular}
\newline
The class \pushcoro provides a mechanism to send a data value from one
execution context to another.\\

\paragraph*{member types\\}
\begin{tabular}{ l l l }
    \midrule

    \cpp{iterator} & std::output\_iterator & (not defined for coroutine<void>::push\_type template specialization)\\

    \midrule
\end{tabular}

\paragraph*{member functions}
\subparagraph*{(constructor)}
constructs new coroutine\\

\begin{tabular}{ l l }
    \midrule

    \cpp{push_type();} & (1)\\

    \midrule

    \cpp{push_type(Function&& fn);} & (2)\\

    \midrule

    \cpp{push_type(push_type&& other);} & (3)\\

    \midrule

    \cpp{push_type(const push_type& other)=delete;} & (4)\\

    \midrule
\end{tabular}

\begin{description}
    \item[1)] creates a \pushcoro which does not represent a context of
              execution
    \item[2)] creates a \pushcoro object and associates it with a execution
              context
    \item[3)] move constructor, constructs a \pushcoro object to represent a
              context of execution that was represented by \textit{other}, after this
              call \textit{other} no longer represents a coroutine
    \item[4)] copy constructor is deleted; coroutines are not copyable\\
\end{description}

{\bf Parameters}
\begin{description}
    \item[other] another coroutine object with which to construct this coroutine object
    \item[fn]    function to execute in the new coroutine\\
\end{description}

{\bf Exceptions}
\begin{description}
    \item[1), 3)] noexcept specification: \cpp{noexcept}
    \item[2)]    \cpp{std::system_error} if the coroutine could not be started
                  - the exception may represent a implementation-specific error
                  condition\\
\end{description}

{\bf Notes}
\newline
If the \corofunction throws an exception, this exception is re-thrown when the caller
returns from\\
\pushcoroop.\\

{\bf Example}
\cppf{access_params.cpp}

\subparagraph*{(destructor)}
destructs a coroutine\\

\begin{tabular}{ l l }
    \midrule

    \cpp{\~push_type();} & (1)\\

    \midrule
\end{tabular}

\begin{description}
    \item[1)] destroys a \pushcoro. If that \pushcoro is associated with a context of execution,
              then the context of execution is destroyed too. Specifically,
              its stack is unwound.\\
\end{description}

\subparagraph*{operator=}
moves the coroutine object\\

\begin{tabular}{ l l }
    \midrule

    \cpp{push_type & operator=(push_type&& other);} & (1)\\

    \midrule

    \cpp{push_type & operator=(const push_type& other)=delete;} & (2)\\

    \midrule
\end{tabular}

\begin{description}
    \item[1)] assigns the state of \textit{other} to *this using move semantics
    \item[2)] copy assignment operator is deleted; coroutines are not copyable\\
\end{description}

{\bf Parameters}
\begin{description}
    \item[other]   another coroutine object to assign to this coroutine object\\
\end{description}

{\bf Return value}
\begin{description}
    \item[*this]
\end{description}

{\bf Exceptions}
\begin{description}
    \item[1)] noexcept specification: \cpp{noexcept}\\
\end{description}

\subparagraph*{operator bool}
indicates if context of execution is still valid, that is, \corofunction has not
finished\\

\begin{tabular}{ l l }
    \midrule

    \cpp{operator bool();} & (1)\\

    \midrule
\end{tabular}

\begin{description}
    \item[1)] evaluates to true if coroutine is not complete (\corofunction has
        not terminated)\\
\end{description}

{\bf Exceptions}
\begin{description}
    \item[1)] noexcept specification: \cpp{noexcept}\\
\end{description}

\subparagraph*{operator()}
jump context of execution\\

\begin{tabular}{ l l l }
    \midrule

    \cpp{push_type & operator()(const Arg& arg);} & (1) & (member of generic template)\\

    \midrule

    \cpp{push_type & operator()(Arg&& arg);} & (2) & (member of generic template)\\

    \midrule

    \cpp{push_type & operator()(Arg& arg);} & (3) & (member only of coroutine<Arg\&>::push\_type\\
                                            &     & template specialization)\\

    \midrule

    \cpp{push_type & operator()();} & (4) & (member only of coroutine<void>::push\_type\\
                                    &     & template specialization)\\

    \midrule
\end{tabular}

\begin{description}
    \item[1),2)] If \textit{Arg} is move-assignable, it will be passed using
        move semantics. Otherwise it will be copied.\\
\end{description}

Switches the context of execution, transferring \textit{arg} to \corofunction.\\

{\bf Note}
\newline
It is important that the coroutine is still valid (\cpp{operator bool()}
returns \cpp{true}) before calling this function, otherwise it results in
undefined behaviour.\\

{\bf Parameters}
\begin{description}
    \item[arg] argument to pass to the \corofunction\\
\end{description}

{\bf Return value}
\begin{description}
    \item[*this]
\end{description}

{\bf Exceptions}
\begin{description}
    \item[1)] \cpp{std::system_error} if control of execution could not be
              transferred to other execution context - the exception may
              represent a implementation-specific error condition; re-throw
              user-defined exceptions from \corofunction\\
\end{description}

\subparagraph*{swap}
swaps two coroutine objects\\

\begin{tabular}{ l l }
    \midrule

    \cpp{void swap(push_type& other);} & (1)\\

    \midrule
\end{tabular}

\begin{description}
    \item[1)] exchanges the underlying context of execution of two coroutine objects\\
\end{description}

{\bf Exceptions}
\begin{description}
    \item[1)] noexcept specification: \cpp{noexcept}\\
\end{description}

\paragraph*{non-member functions}
\subparagraph*{std::swap}
Specializes \cpp{std::swap} for \pushcoro and swaps the underlying context of
lhs and rhs.\\

\begin{tabular}{ l l }
    \midrule

    \cpp{void swap(push_type& lhs,push_type& rhs);} & (1)\\

    \midrule
\end{tabular}

\begin{description}
    \item[1)] exchanges the underlying context of execution of two coroutine
              objects by calling \cpp{lhs.swap(rhs)}.\\
\end{description}

{\bf Exceptions}
\begin{description}
    \item[1)] noexcept specification: \cpp{noexcept}\\
\end{description}

\subparagraph*{std::begin}
Specializes \cpp{std::begin} for \pushcoro.\\

\begin{tabular}{ l l }
    \midrule

    \cpp{template<class R> coroutine<R>::push_type::iterator begin(coroutine<R>::push_type& c);} & (1)\\

    \midrule
\end{tabular}

\begin{description}
    \item[1)] creates and returns a \cpp{std::output_iterator}\\
\end{description}

\subparagraph*{std::end}
Specializes \cpp{std::end} for \pushcoro.\\

\begin{tabular}{ l l }
    \midrule

    \cpp{template<class R> coroutine<R>::push_type::iterator end(coroutine<R>::push_type& c);} & (1)\\

    \midrule
\end{tabular}

\begin{description}
    \item[1)] creates and returns a \cpp{std::output_iterator} indicating the termination of the coroutine\\
\end{description}

Calling \cpp{iterator::operator*(Arg&&)} switches the execution context and transfers the given data value.\\
\cpp{iterator::operator*(Arg&&)} returns if other context has transferred control of execution back.\\
The iterator is forward-only.\\

{\bf Example}
\cppf{output_iterator.cpp}


\subsubsection*{std::coroutine\_errc}
Defined in header \cpp{<coroutine>}.\\

\begin{tabular}{ l }
    \midrule

    \cpp{enum class coroutine_errc \{ no_data \};}\\

    \midrule
\end{tabular}

Enumeration \cpp{std::coroutine_errc} defines the error codes reported by
\pullcoro in\\
\cpp{std::coroutine_error} exception object.

\paragraph*{member constants}
Determines error code.\\

\begin{tabular}{ l l }
    \midrule

    \cpp{no_data} & \pullcoro has no valid data (maybe moved by prior access)\\

    \midrule
\end{tabular}
\newline


\subsubsection*{std::coroutine\_error}
Defined in header \cpp{<coroutine>}.\\

\begin{tabular}{ l }
    \midrule

    \cpp{class coroutine_error;}\\

    \midrule
\end{tabular}

The class \cpp{std::coroutine_error} defines an exception class that is derived
from \cpp{std::logic_error}.

\paragraph*{member functions}
\subparagraph*{(constructor)}
constructs new coroutine error object.\\

\begin{tabular}{ l l }
    \midrule

    \cpp{coroutine_error( std::error_code ec);} & (1)\\

    \midrule
\end{tabular}

\begin{description}
    \item[1)] creates a \cpp{std::coroutine_error} error object from an error-code.\\
\end{description}

{\bf Parameters}
\begin{description}
    \item[ec] error-code
\end{description}

\subparagraph*{code}
Returns the error-code.\\

\begin{tabular}{ l l }
    \midrule

    \cpp{const std::error_code& code() const;} & (1)\\

    \midrule
\end{tabular}

\begin{description}
    \item[1)] returns the stored error code.\\
\end{description}

{\bf Return value}
\begin{description}
    \item[std::error\_code] stored error code\\
\end{description}

{\bf Exceptions}
\begin{description}
    \item[1)] noexcept specification: \cpp{noexcept}\\
\end{description}

\subparagraph*{what}
Returns a error-description.\\

\begin{tabular}{ l l }
    \midrule

    \cpp{virtual const char* what() const;} & (1)\\

    \midrule
\end{tabular}

\begin{description}
    \item[1)] returns a description of the error.\\
\end{description}

{\bf Return value}
\begin{description}
    \item[char*] null-terminated string with error description\\
\end{description}

{\bf Exceptions}
\begin{description}
    \item[1)] noexcept specification: \cpp{noexcept}\\
\end{description}


%//////////////////////////////////////////////////////////////////////////////

\newpage
\addcontentsline{toc}{subsection}{References}
\begin{thebibliography}{99}

    \bibitem{N3985}
        \href{http://www.open-std.org/jtc1/sc22/wg21/docs/papers/2014/n3985.pdf}
        {N3985: A proposal to add coroutines to the C++ standard library, Revision 1}

    \bibitem{N4024}
        \href{http://www.open-std.org/jtc1/sc22/wg21/docs/papers/2014/n4024.pdf}
        {N4024: Distinguishing coroutines and fibers}

    \bibitem{N4134}
        \href{http://www.open-std.org/jtc1/sc22/wg21/docs/papers/2014/n4134.pdf}
        {N4134: Resumable Functions v.2}

    \bibitem{N4244}
        \href{http://www.open-std.org/jtc1/sc22/wg21/docs/papers/2014/n4244.pdf}
        {N4244: Resumable Lambdas}

    \bibitem{N4398}
        \href{http://www.open-std.org/jtc1/sc22/wg21/docs/papers/2015/n4398.pdf}
        {N4398: A unified syntax for stackless and stackful coroutines}

    \bibitem{gccsplit}
        \href{http://gcc.gnu.org/wiki/SplitStacks}
        {Split Stacks / GCC}

    \bibitem{llvmseg}
        \href{http://llvm.org/releases/3.0/docs/SegmentedStacks.html}
        {Segmented Stacks / LLVM}

    \bibitem{bcontext}
        Library \emph{Boost.Context}:
        \href{https://github.com/boostorg/context} {git repo},
        \href{http://www.boost.org/doc/libs/1_58_0/libs/context/doc/html/index.html} {documentation}

    \bibitem{bcoroutine2}
        Library \emph{Boost.Coroutine2}:
        \href{https://github.com/boostorg/coroutine2} {git repo},
        \href{http://olk.github.io/libs/coroutine2/doc/html/index.html} {documentation}

    \bibitem{bfiber}
        Library \emph{Boost.Fiber}:
        \href{https://github.com/olk/boost-fiber} {git repo},
        \href{http://olk.github.io/libs/fiber/doc/html/index.html} {documentation}

    \bibitem{cactusstack}
        \emph{cactus stack}:
        \href{http://c2.com/cgi/wiki?CactusStack} {cactus stack},
        \href{http://en.wikipedia.org/wiki/Parent_pointer_tree} {parent pointer tree}

\end{thebibliography}


%//////////////////////////////////////////////////////////////////////////////

\abschnitt{A. x86\_64 SYSV calling convention}

\asmf{foo.S}

\asmf{bar.S}


%//////////////////////////////////////////////////////////////////////////////

\end{document}
